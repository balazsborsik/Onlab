\documentclass[12pt,a4paper]{article}
\usepackage[utf8]{inputenc}
\usepackage{t1enc}
\usepackage[magyar]{babel}
\usepackage{graphicx}
\usepackage{geometry}
\usepackage{amssymb}
\usepackage{amsmath}
\usepackage{algorithm}
\usepackage{algpseudocode}
 \geometry{ a4paper, textwidth=180mm, textheight=257mm, left=15mm, top=20mm, bottom=25mm}
\usepackage{fancyhdr}
\pagestyle{fancy}

\chead{\includegraphics[height=15mm]{bme_logo_kicsi}}
\setlength{\headsep}{50pt}
\geometry{bottom=30mm}
\fancyfoot[C]{\footnotesize\sl
\begin{tabular}{r|l} 
Budapesti Műszaki és Gazdaságtudományi Egyetem & 1117 Budapest, Magyar tudósok körútja 2. I.B.132.\\
Villamosmérnöki és Informatikai Kar & Telefon: 463-2585, Fax: 463-3157\\
Számítástudományi és Információelméleti Tanszék & http:\textbackslash\textbackslash www.cs.bme.hu\\
\end{tabular}}
\renewcommand{\headrulewidth}{0.4pt}\renewcommand{\footrulewidth}{0.4pt}

\begin{document} \ \
\vspace{2mm}
\begin{center}
{\Large\sc
Önálló Labor\\[5mm]
\bf 
Random konstrukciók programozása a Zarankiewicz problémához
\\[10mm]
{\Large
Borsik Balázs
}\\[10mm]
Témavezető: Dr.Héger Tamás
}
\end{center}
\vspace{1cm}

\section*{1. Alapfogalmak}
\textbf{Páros gráf:} (Szükséges-e? – Nem biztos.)\\[2mm]
\textbf{Zarankiewicz-szám:} Egy $G = (A,B;E)$ páros gráf $K_{s,t}$-mentes, ha nem tartalmaz olyan $s$ csúcsot $A$-ban és $t$ csúcsot $B$-ben, amelyek egy $K_{s,t}$-vel izomorf részgráfot alkotnak. Az $(m,n)$ méretű $K_{s,t}$-mentes páros gráf éleinek maximális számát $Z_{s,t}(m,n)$-nel jelöljük, és Zarankiewicz-számnak nevezzük.\\[2mm]
\textbf{Zarankiewicz-probléma (eredeti megfogalmazás):} Vegyünk egy $n \times m$-es mátrixot, amely csak 0 és 1 értékeket tartalmaz. Legkevesebb hány darab 1-es szükséges, hogy mindenképp tartalmazzon $s \times t$-es csupa 1-es részmátrixot? Ez a definíció megegyezik $Z_{s,t}(m,n) + 1$-gyel.

\section*{2. Bemutatás}
Mint látható, a Zarankiewicz probléma egy extremális gráfelméleti probléma. A problémára több képlet is ad értelmezhet felső becsléseket, de ezek a becslések általában a paraméterek nagyságának vagy aszimmetriájának növelésével jelentősen eltérhetnek a pontos értékektől.\\

Alsó becslésekre kevesebb képlet található; ezekben az esetekben különféle konstrukciók (pl. projektív síkok, affin síkok), vagy kimerítő keresések adhatnak jobb közelítő értéket.\\

Az önálló labor során a random konstrukciókkal kapott alsó becslések vizsgálata és programozása történt.

\section*{3. Felső becslések bemutatása}
A felső becslésekhez alkalmazható például a \textbf{Jensen-egyenlőtlenség}, amely az alábbi módon fogalmazható meg:

\[
\phi\left( \frac{1}{n} \sum_{i=1}^{n} x_i \right) \leq \frac{1}{n} \sum_{i=1}^{n} \phi(x_i),
\]

ha $\phi$ konvex függvény. Ez az egyenlőtlenség segít az egyes Zarankiewicz-számokra adott becslések levezetésében.
\\[2mm]
A projekt során is használt felső becslés pedig \textbf{Roman becslése} volt, ami a Jensen-egyenlőtlenségből kapott paraméterezett változatt, így sokkal kényelmesebben használható. A program során ez a becslés adta az élvalószínűségek alapját.
\\[2mm]
\textbf{Roman-egyenlőtlenség:} Legyen $I \subset \mathbb{R}$ egy intervallum, $f : I \to \mathbb{R}$ egy szigorúan növekvő konvex vagy szigorúan csökkenő konkáv függvény, $n \in \mathbb{N}$, $x_1, \dots, x_n, p, p+1 \in I \cap \mathbb{Z}$. Ekkor
\[
\sum_{i=1}^{n} x_i \leq \frac{\sum_{i=1}^{n} f(x_i)}{f(p+1) - f(p)} + n \cdot \frac{p f(p+1) - (p+1) f(p)}{f(p+1) - f(p)}.
\]
\\[2mm]
Az egyenlőség pontosan akkor áll fenn, ha $x_i \in \{p, p+1\}$ minden $1 \leq i \leq n$ esetén, vagy ha $\{x_1, \dots, x_n, p, p+1\} \subset I'$ egy olyan intervallumban, amelyen $f$ lineáris.
\\[2mm]
\textbf{Roman-féle felső korlát:} Legyen $G = (A, B; E)$ egy $K_{s,t}$-mentes páros gráf, ahol $|A| = m$, $|B| = n$, és $p \geq s - 1$. Ekkor $G$ éleinek számára teljesül:
\[
e(G) \leq \frac{t - 1}{\binom{p}{s - 1}} \binom{m}{s} + n \cdot \frac{(p + 1)(s - 1)}{s}.
\]
\\[2mm]

\begin{algorithm}
\caption{Moser Tardos pszeudokódja}
\begin{algorithmic}[1]

\Function{\textproc{BecsültMaxÉlek}}{$m$, $n$, $\text{ismétlések}$}
    \State $\text{maxélek} \gets 0$
    \State $\text{élek} \gets$ minden lehetséges él $m$ és $n$ között
    \State $p \gets$ felső becslés alapján számolt valószínűség

    \For{$i = 1$ \textbf{tól} $\text{ismétlések}$}
        \State gráf inicializálása üres $m \times n$ mátrixként
        \State véletlenszerű élek beszúrása $p$ valószínűséggel

        \While{van benne $K_{2,2}$}
            \State távolítsuk el azt az $\text{élt}$, amit a legtöbb $K_{2,2}$ tartalmaz
        \EndWhile

        \State véletlenszerűen keverjük az éleket

        \For{minden él}
            \If{nem hoz létre $K_{2,2}$-t}
                \State adjuk hozzá a gráfhoz
            \EndIf
        \EndFor

        \If{aktuális gráf él száma $>$ $\text{maxélek}$}
            \State $\text{maxélek} \gets$ új érték
        \EndIf
    \EndFor

    \State \Return $\text{maxélek}$
\EndFunction

\end{algorithmic}
\end{algorithm}


\vspace{1cm}\vfill
Budapest, \today
\begin{flushright}
\begin{minipage}{0.5\textwidth}
\begin{center}
	dr. Katona Gyula\\
	\textit{egyetemi docens}\\
	\textit{tanszékvezető}
\end{center}
\end{minipage}
\end{flushright}
\vfill
\end{document}
